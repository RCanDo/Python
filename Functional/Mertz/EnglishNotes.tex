\newpage
\section{English Notes}
\subsection{Luck-Effort dilemma}
15.03.2013

Q: What is our life-success dependent on: luck or effort?

Perhaps it is mainly a matter of perception
but let us try to construct some simple model describing how this perception
transforms into reality.

Suppose that common perception (in some population of individuals)
is that success is a result of effort and dedication.
One might expect that such perception correlates with a competition-oriented
attitude to life.
Thus common effort-perception results in competitive population
in which essentially one fight with each other.
But this leads to very tough and unpredictable environment in which
individuals' performance (in terms of their success or well-being)
depends heavily on their skills and many circumstances they are not able to control.
In effect we obtain environment in which rather luck matters then effort
(unless one is outstandingly strong to overcome all the adversities).
In short: common effort-perception results in volatile environment.

But the reverse is also true: luck-perception gives more stable-environment
in which effort and dedication are more effectively utilized.
It is so because luck-perception should result in more cooperative attitude
to life.
In cooperative society randomness and volatility of environment are more likely to be mitigated

Thus we came to contradiction, which however should not be considered
as undermining sensibility of this reasoning, but rather as a starting point
for constructing dynamic model with negative feedback.

So the final question is if the luck-effort perception really correlates with
the competitive-cooperative attitude.
We based above reasoning on  the assumption that luck-perception results in cooperative attitude while effort-p. turns into competitive attitude.
Is it really true?
Maybe reverse is true?
Or maybe none - there is no correlation at all?


